%!TEX root = ../report.tex
\clearpage
\chapter{Introduction}
\label{ch:introduction}
This document presents architecture recovery of
Docker\footnote{\url{https://www.docker.com/}} by identifying software patterns
and performing an evaluation of the architecture based on the identified
patterns. This document is part of the Software Pattern assignment at the
University of Groningen.

Docker is an open-source project that automates the deployment of applications
inside software containers, by providing an additional layer of abstraction and
automation of operating-system-level virtualization on Linux \cite{dockerdef}.

The structure of the document is explained as follows. Chapter \ref{ch:context}
gives brief explanation with regard to Docker. Chapter \ref{ch:stakeholders}
elaborates on the stakeholders involved in the Docker project and its
corresponding key-drivers. Patterns discovered in the Docker project are
documented in chapter \ref{ch:patterns}. Evaluation is presented in chapter
\ref{ch:evaluation}. Chapter \ref{ch:recommendations} gives several
recommendation for the Docker project. Lastly, a conclusion is drawn in chapter
\ref{ch:conclusion}.

This project utilizes the IDAPO\footnote{Identifying Architectural Patterns in
Open Source Software} process to recover the architecture \cite{idapo}. The
PBAR\footnote{Pattern-Based Architecture Reviews} approach is used to perform
the evaluation \cite{pbar}.

There are 12 steps in IDAPO. Those steps are (1) identifying the type and domain
of the product, (2) identifying used technologies, (3) studying used
technologies, (4) identifying candidate patterns, (5) reading patterns
literature, (6) studying the documentation, (7) studying the source code, (8)
studying components \& connectors, (9) identifying patterns and variants, (10)
validating identified patterns, (11) getting the feedback from community, and
(12) register pattern usage.

Chapter \ref{ch:context} and Chapter \ref{ch:stakeholders} correspond step (1)
of IDAPO. Step (2) to (5) are related to Chapter \ref{ch:softwarearch}, where
used technologies are presented in the architecture. Chapter \ref{ch:patterns}
corresponds to step (6) to (12), where the result of those steps is the pattern
documentation.

While IDAPO recovers the patterns inside Docker, PBAR process determines the
high-level process of pattern-based recovery and the evaluation of the
architecture. The evaluation is documented in Chapter \ref{ch:evaluation}.